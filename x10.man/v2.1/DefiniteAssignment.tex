\chapter{Definite Assignment}
\label{sect:DefiniteAssignment}
\index{definite assignment}
\index{assignment!definite}
\index{definitely assigned}
\index{definitely not assigned}

X10 requires, reasonably enough, that every variable be set before it is read.
Sometimes this is easy, as when a variable is declared and assigned together: 
%~~gen ^^^ DefiniteAssignment4x1u
% package DefiniteAssignment4x1u;
% class Example {
% def example() {
%~~vis
\begin{xten}
  var x : Int = 0;
  assert x == 0;
\end{xten}
%~~siv
%}}
%~~neg
However, it is convenient to allow programs to make decisions before
initializing variables.
%~~gen ^^^ DefiniteAssignment4u7z
% package DefiniteAssignment4u7z;
% class Example {
%~~vis
\begin{xten}
def example(a:Int, b:Int) {
  val max:Int;
  // max cannot be read here.
  if (a > b) max = a;
  else max = b;
  assert max >= a && max >= b;
}
\end{xten}
%~~siv
%}
%~~neg
This is particularly useful for \xcd`val` variables.  \xcd`var`s could be
initialized to a default value and then reassigned with the right value, but
\xcd`val`s must be initialized once correctly and cannot be changed. 

However, one must be careful -- and the X10 compiler enforces this care.
Without the \xcd`else` clause, the preceding code might not give \xcd`max` a
value by the \xcd`assert`.  

This leads to the concept of {\em definite assignment} \cite{Javasomething}.
A variable is definitely assigned at a point in code if, no matter how that
point in code is reached, the variable has been assigned to.  In X10,
variables must be definitely assigned before they can be read.


As X10 requires that \xcd`val` variables {\em not} be initialized
twice,  we need the dual concept as well.  A variable is {\em definitely
unassigned} at a point in code if it cannot have been assigned there.  For
example, immediately after \xcd`val x:Int`, \xcd`x` is definitely unassigned.  

At some points in code, a variable might be neither definitely assigned nor
definitely unassigned.    Such states are not always useful.  
%~~gen ^^^ DefiniteAssignment4f5z
% package DefiniteAssignment4f5z;
% class Example {
% 
%~~vis
\begin{xten}
def example(flag : Boolean) {
  var x : Int;
  if (flag) x = 1;
  // x is neither def. assigned nor unassigned.
  x = 2; 
  // x is def. assigned.
\end{xten}
%~~siv
% } } 
%~~neg
This shows that we cannot simply define ``definitely unassigned'' as ``not
definitely assigned''.   If \xcd`x` had been a \xcd`val` rather than a
\xcd`var`, the previous example would not be allowed.  

Unfortunately, a completely accurate definition of ``definitely assigned''
or ``definitely unassigned'' is undecidable -- impossible for the compiler to
determine.  So, X10 takes a {\em conservative approximation} of these
concepts.  If X10's definition says that \xcd`x` is definitely assigned (or
definitely unassigned), then it will be assigned (or not assigned) in every
execution of the program.  

However, there are programs which X10's algorithm says are incorrect, but
which actually would behave properly if they were executed.   In the following
example, \xcd`flag` is either \xcd`true` or \xcd`false`, and in either case
\xcd`x` will be initialized.  However, X10's analysis does not understand this
--- thought it {\em would} understand if the example were coded with an
\xcd`if-else` rather than a pair of \xcd`if`s.  So, after the two \xcd`if`
statements, \xcd`x` is not definitely assigned, and thus the \xcd`assert`
statement, which reads it, is forbidden.  
%~~gen ^^^ DefiniteAssignment3x6i
% package DefiniteAssignment3x6i;
% class Example{ 
%~~vis
\begin{xten}
def example(flag:Boolean) {
  var x : Int;
  if (flag) x = 1;
  if (!flag) x = 2;
  // Not Allowed: assert x < 3;
}
\end{xten}
%~~siv
%}
%~~neg

\section{Technical Details of Definite Assignment}

The material in this section is heavily based on \cite{Java-lang-spec}.

To describe definite assignment, we need several concepts.  In this, let $v$
be a local variable, and $c$ be a statement or expression of X10.  We will define
the following: 
\begin{itemize}
\item $v$ is {\em definitely assigned before } $c$;
\item $v$ is {\em definitely assigned after } $c$;
\item $v$ is {\em definitely assigned after true } $c$, when $c$ is a Boolean expression;
\item $v$ is {\em definitely assigned after false } $c$, when $c$ is a Boolean expression.
\end{itemize}

The intent of the phrase ``$v$ is definitely assigned after $c$'' is that, if
$c$ completes normally, $v$ will be assigned.  There is no requirement on what
happens if $c$ does not complete normally.  For example, \xcd`throw E` never
completes normally -- and so ``$v$ is definitely assigned after 
\xcd`throw E`'' should be (and will be) true, even though a \xcd`throw` does not assign
to $v$. 

The intent of ``$v$ is definitely assigned after true $c$'' is that, if $c$
evaluates to \xcd`true`, then afterwards $v$ will be assigned.  There is no
requirement on $v$ if $c$ terminates abnormally, or if it evaluates to
\xcd`false`.  So, ``$v$ is definitely assigned after true \xcd`false`'' should
be (and will be) true, because \xcd`false` is never true.

The definition of the ``definitely assigned'' constructs is done by analysis
of the structure of the command $c$.

\paragraph{Boolean Expressions}

\begin{itemize}
\item $v$ is definitely assigned after true \xcd`false`.
      (Rationale: this is vacuously the case, as \xcd`false` is never true.) 
\item $v$ is definitely assigned after false \xcd`true`.  
\item Otherwise, if $c$ is a literal or variable Boolean expression, 
      then  $v$ is definitely assigned before $c$ iff $v$ is definitely
      assigned after $c$.
\item Otherwise, $v$ is definitely assigned before $c$ iff 
      both $v$ is definitely assigned before true $c$ 
      and $v$ is definitely assigned before false $c$.
      (Rationale: If $c$ terminates normally, it will either be true or false,
      and in either case, $v$ will be assigned.)
\end{itemize}

The remaining clauses define the ``definitely assigned before true'' and
``definitely assigned before false'' concepts.

\paragraph{{\tt \&\&}}
\begin{itemize}
\item $v$ is definitely assigned after true \xcdmath"$c$ && $d$" iff 
      $v$ is definitely assigned after true \xcdmath"$c$" or   
      $v$ is definitely assigned after true \xcdmath"$d$".
      (Rationale: if \xcdmath"$c$ && $d$" evaluates to true, then both $c$ and
      $d$ evalute to true.  If either one of these definitely assigns $v$,
      then the conjunction  \xcdmath"$c$ && $d$" definitely assigns $v$ as
      well, and vice-versa.  The other clauses for \xcd`&&` and \xcd`||` are similar.)
\item $v$ is definitely assigned after false \xcdmath"$c$ && $d$" iff 
      $v$ is definitely assigned after false \xcdmath"$c$" and
      $v$ is definitely assigned after false \xcdmath"$d$".
      
\item $v$ is definitely assigned before $c$ iff 
      $v$ is definitely assigned before \xcdmath"$c$ && $d$".

\item $v$ is definitely assigned before $d$ iff 
      $v$ is definitely assigned after true $c$.  

\end{itemize}

\paragraph{{\tt ||}}
\begin{itemize}
\item $v$ is definitely assigned after true \xcdmath"$c$ || $d$" iff 
      $v$ is definitely assigned after true \xcdmath"$c$" and 
      $v$ is definitely assigned after true \xcdmath"$d$".

\item $v$ is definitely assigned after false \xcdmath"$c$ || $d$" iff 
      $v$ is definitely assigned after false \xcdmath"$c$" or 
      $v$ is definitely assigned after false \xcdmath"$d$".
      
\item $v$ is definitely assigned before $c$ iff 
      $v$ is definitely assigned before \xcdmath"$c$ || $d$".

\item $v$ is definitely assigned before $d$ iff 
      $v$ is definitely assigned after false $c$.  

\end{itemize}

\paragraph{{\tt !} on Booleans}

\begin{itemize}
\item $v$ is definitely assigned after true \xcdmath"!$c$" 
      iff 
       $v$ is definitely assigned after false \xcdmath"c".
\item $v$ is definitely assigned after false \xcdmath"!$c$" 
      iff 
       $v$ is definitely assigned after true \xcdmath"c".
\item  $v$ is definitely assigned before \xcdmath"!$c$"  
       iff
       $v$ is definitely assigned before \xcdmath"$c$".
\end{itemize}

\paragraph{{\tt \&} on Booleans}
\begin{itemize}
\item  $v$ is definitely assigned after true \xcdmath"$c$ & $d$" 
        iff 
         $v$ is definitely assigned after true \xcdmath"$c$" or 
         $v$ is definitely assigned after true \xcdmath"$d$".
\item  $v$ is definitely assigned after false \xcdmath"$c$ & $d$" 
iff
either  $v$ is definitely assigned after \xcdmath"$d$" 
or  $v$ is definitely assigned after false \xcdmath"$c$" and  $v$ is
definitely assigned after false \xcdmath"$d$".
\item  $v$ is definitely assigned before  \xcdmath"$c$ & $d$" 
       iff  $v$ is definitely assigned before \xcdmath"$c$" 
\item  $v$ is definitely assigned before \xcdmath"$d$" 
       iff 
       $v$ is definitely assigned after \xcdmath"$c$".
\end{itemize}

\paragraph{{\tt |} on Booleans}
\begin{itemize}
\item  $v$ is definitely assigned after false \xcdmath"$c$ | $d$" 
        iff 
         $v$ is definitely assigned after false \xcdmath"$c$" and
         $v$ is definitely assigned after false \xcdmath"$d$".
\item  $v$ is definitely assigned after true \xcdmath"$c$ | $d$" 
iff
either  $v$ is definitely assigned after \xcdmath"$d$" 
or  $v$ is definitely assigned after true \xcdmath"$c$" and  $v$ is
definitely assigned after true \xcdmath"$d$".
\item  $v$ is definitely assigned before  \xcdmath"$c$ | $d$" 
       iff  $v$ is definitely assigned before \xcdmath"$c$" 
\item  $v$ is definitely assigned before \xcdmath"$d$" 
       iff 
       $v$ is definitely assigned after \xcdmath"$c$".
\end{itemize}

\paragraph{{\tt \^{~}} on Booleans}
Since \xcdmath"$c$ != $d$" is identical to \xcdmath"$c$^$d$", these rules
apply to \xcd`!=` as well.
\begin{itemize}
\item  $v$ is definitely assigned after true \xcdmath"$c$^$d$" in one of the
       following three cases: 
       (1)  $v$ is definitely assigned after $d$;
       (2)  $v$ is definitely assigned after true $c$ and  $v$ is
            definitely assigned after true $d$;
       (3)  $v$ is definitely assigned after false $c$ and  $v$ is
            definitely assigned after false $d$.

\item  $v$ is definitely assigned after false \xcdmath"$c$^$d$" in one of the
       following three cases: 
       (1)  $v$ is definitely assigned after $d$;
       (2)  $v$ is definitely assigned after false $c$ and  $v$ is
            definitely assigned after true $d$;
       (3)  $v$ is definitely assigned after true $c$ and  $v$ is
            definitely assigned after false $d$.
\item  $v$ is definitely assigned before  \xcdmath"$c$ ^ $d$" 
       iff  $v$ is definitely assigned before \xcdmath"$c$" 
\item  $v$ is definitely assigned before \xcdmath"$d$" 
       iff 
       $v$ is definitely assigned after \xcdmath"$c$".

\end{itemize}

\paragraph{{\tt ==} on Booleans}
\begin{itemize}
\item  $v$ is definitely assigned after true \xcdmath"$c$==$d$" in one of the
       following three cases: 
       (1)  $v$ is definitely assigned after $d$;
       (2)  $v$ is definitely assigned after true $c$ and  $v$ is
            definitely assigned after false $d$;
       (3)  $v$ is definitely assigned after false $c$ and  $v$ is
            definitely assigned after true $d$.

\item  $v$ is definitely assigned after false \xcdmath"$c$==$d$" in one of the
       following three cases: 
       (1)  $v$ is definitely assigned after $d$;
       (2)  $v$ is definitely assigned after true $c$ and  $v$ is
            definitely assigned after true $d$;
       (3)  $v$ is definitely assigned after false $c$ and  $v$ is
            definitely assigned after false $d$.
\item  $v$ is definitely assigned before  \xcdmath"$c$ == $d$" 
       iff  $v$ is definitely assigned before \xcdmath"$c$" 
\item  $v$ is definitely assigned before \xcdmath"$d$" 
       iff 
       $v$ is definitely assigned after \xcdmath"$c$".

\end{itemize}

\paragraph{{\tt ? :} on Booleans}
If $t$, $c$, and $d$ are Boolean-valued expressions, so is 
\xcdmath"$t$ ? $c$ : $d$", 
leading to the following rules: 
\begin{itemize}
\item  $v$ is definitely assigned after true \xcdmath"$t$ ? $c$ : $d$" 
       iff both
       ( $v$ is definitely assigned before $c$ or  $v$ is definitely assigned
       after true $c$ )
       and
       ( $v$ is definitely assigned before $d$ or  $v$ is definitely assigned
       after true $d$ )

\item  $v$ is definitely assigned after false \xcdmath"$t$ ? $c$ : $d$" 
       iff both
       ( $v$ is definitely assigned before $c$ or  $v$ is definitely assigned
       after false $c$ )
       and
       ( $v$ is definitely assigned before $d$ or  $v$ is definitely assigned
       after false $d$ )
\item   $v$ is definitely assigned before \xcdmath"$t$ ? $c$ : $d$" 
       iff  $v$ is definitely assigned before $t$;
\item $v$ is definitely assigned before $c$ iff  $v$ is definitely assigned
      after true $t$.
\item $v$ is definitely assigned before $d$ iff  $v$ is definitely assigned
      after false $t$.
\end{itemize}


\paragraph{non-Boolean {\tt ? :}}
If 
$c$ and $d$ are non-Boolean-valued expressions, so is 
\xcdmath"$t$ ? $c$ : $d$", leading to the following rules: 
\begin{itemize}
\item  $v$ is definitely assigned after \xcdmath"$t$ ? $c$ : $d$"
       iff  $v$ is definitely assigned after $c$ and $v$ is definitely
       assigned after $d$.
\item   $v$ is definitely assigned before \xcdmath"$t$ ? $c$ : $d$" 
       iff  $v$ is definitely assigned before $t$;
\item $v$ is definitely assigned before $c$ iff  $v$ is definitely assigned
      after true $t$.
\item $v$ is definitely assigned before $d$ iff  $v$ is definitely assigned
      after false $t$.
\end{itemize}

 
 \paragraph{Assignments at Boolean type}
 
 If a statement $s$ has the form 
  \xcdmath"$c$=$d$",
  \xcdmath"$c$ &=$d$",
  \xcdmath"$c$ |=$d$", or  
  \xcdmath"$c$ ^=$d$", and is 
 is Boolean-valued, then: 
 \begin{itemize}
 \item  $v$ is definitely assigned before $c$ iff $v$ is definitely assigned
        before $s$;
 \item $v$ is definitely assigned before $d$ iff $v$ is definitely assigned
       after $c$;
 \item $v$ is definitely assigned after true \xcdmath"$c$=$d$" iff either $v=c$
       or $v$ is definitely assigned after true $d$;
 \item $v$ is definitely assigned after false \xcdmath"$c$=$d$" iff either $v=c$
       or $v$ is definitely assigned after false $d$;
 
 \item $v$ is definitely assigned after true \xcdmath"$c$&=$d$" iff either $v=c$
       or $v$ is definitely assigned after true \xcdmath"$c$ & $d$"
 \item $v$ is definitely assigned after false \xcdmath"$c$&=$d$" iff either $v=c$
       or $v$ is definitely assigned after false \xcdmath"$c$ & $d$"
 
 \item $v$ is definitely assigned after true \xcdmath"$c$|=$d$" iff either $v=c$
       or $v$ is definitely assigned after true \xcdmath"$c$ | $d$"
 \item $v$ is definitely assigned after false \xcdmath"$c$|=$d$" iff either $v=c$
       or $v$ is definitely assigned after false \xcdmath"$c$ | $d$"
 

 \item $v$ is definitely assigned after true \xcdmath"$c$^=$d$" iff either $v=c$
       or $v$ is definitely assigned after true \xcdmath"$c$ ^ $d$"
 \item $v$ is definitely assigned after false \xcdmath"$c$^=$d$" iff either $v=c$
       or $v$ is definitely assigned after false \xcdmath"$c$ ^ $d$"
 
 \end{itemize}

\paragraph{Other Assignments}

Suppose that $s$ has the form 
\xcdmath"$c$ = $d$", or 
\xcdmath"$c$ $\star$= $d$" for some binary operator $\star$, where $d$'s type
is not Boolean.

\begin{itemize}
\item $v$ is definitely assigned after $s$ iff either $c=v$ or $v$ is
      definitely assigned after $d$.
\item $v$ is definitely assigned before $c$ iff $v$ is definitely assigned
      before $s$.
\item $v$ is definitely assigned before $d$ iff $v$ is definitely assigned
      after $c$.
\end{itemize}

\paragraph{{\tt ++ } and {\tt --}}

Let $s$ be a pre- or post- increment or decrement expression of expression $c$.
$v$ is definitely assigned after $s$ iff either $v=c$ or $v$ is definitely
assigned after $c$.  $v$ is definitely assigned before $c$ iff $v$ is
definitely assigned before $s$.

\paragraph{Other Expressions}

If an expression $s$ is not boolean-valued and is not a conditional-operator expression or assignment expression, the following rules apply:

\begin{itemize}
\item If $s$ has no subexpressions, $v$ is definitely assigned after $s$ iff
      $v$ is definitely assigned before $s$. This case applies to literals,
      simple variables, \xcd`this`, \xcd`super`, \xcd`null`, \xcd`self`, and
      \xcd`here`. 
\item 
If $s$ has subexpressions, $v$ is definitely assigned after $s$ iff $v$ is definitely assigned after its rightmost immediate subexpression. 
\end{itemize}

For any immediate subexpression $y$ of an expression $x$, $v$ is definitely assigned before $y$ iff $v$ is definitely assigned before $x$ or one of the following situations is true:

\begin{itemize}
    \item $y$ is the right-hand operand of a binary operator and $v$ is definitely assigned after the left-hand operand.
    \item $x$ is a method invocation expression with receiver $r$, $y$ is the first argument
          expression in the method invocation expression; and $v$ is
          definitely assigned after $r$;
    \item $x$ is a method invocation
 expression or class instance creation expression; $y$ is an argument expression, but not the first; and $v$ is definitely assigned after the argument expression to the left of $y$.
\end{itemize}
\noo{What other expressions do we have?  Array constructors?}

\subsection{Definite Assignment and Statements}


\paragraph{Empty Statements}

$v$ is definitely assigned after an empty statement if  $v$ is definitely assigned
before the empty statement.

\paragraph{Blocks}

\begin{itemize}
    \item $v$ is definitely assigned after an empty block iff it is definitely assigned before the empty block.
    \item $v$ is definitely assigned after a nonempty block iff it is definitely assigned after the last statement in the block.
    \item $v$ is definitely assigned before the first statement of the block iff it is definitely assigned before the block.
    \item $v$ is definitely assigned before any other statement $s$ of the
          block iff it is definitely assigned after the statement immediately
          preceding $s$ in the block. 
\end{itemize}

\paragraph{Local Variable Declaration Statements}

Let $s$ be a local variable declaration statement.  Note that $s$ may declare
many variables, and may or may not have initialization expressions.

\begin{itemize}
    \item $v$ is definitely assigned after a local variable declaration statement that contains no initializers iff it is definitely assigned before the local variable declaration statement.
    \item $v$ is definitely assigned after a local variable declaration statement that contains initializers iff either it is definitely assigned after the last initializer expression in the local variable declaration statement or the last initializer expression in the declaration is in the declarator that declares $v$.
    \item $v$ is definitely assigned before the first initializer expression iff it is definitely assigned before the local variable declaration statement.
    \item $v$ is definitely assigned before any other initializer expression e iff either it is definitely assigned after the initializer expression immediately preceding e in the local variable declaration statement or the initializer expression immediately preceding e in the local variable declaration statement is in the declarator that declares $v$. 
\end{itemize}

\paragraph{Labeled Statements}
\begin{itemize}
    \item $v$ is definitely assigned after a labeled statement \xcdmath"$L$:$s$" (where $L$ is a label) iff $v$ is definitely assigned after $s$ and $v$ is definitely assigned before every \xcd`break` statement that may exit \xcdmath"$L$:$s$"
    \item $v$ is definitely assigned before $s$ iff $v$ is definitely assigned before \xcdmath"$L$:$s$".\end{itemize}

\paragraph{ Expression Statements}

\begin{itemize}    \item $v$ is definitely assigned after an expression statement \xcdmath"$e$;" iff it is definitely assigned after $e$.
    \item $v$ is definitely assigned before $e$ iff it is definitely assigned before \xcdmath"$e$;"
\end{itemize}

\paragraph{Conditionals}

\begin{itemize}    
\item $v$ is definitely assigned after \xcdmath"if ($e$) $s$" iff $v$ is definitely assigned after $s$ and $v$ is definitely assigned after $e$ when false.
    \item $v$ is definitely assigned before $e$ iff $v$ is definitely assigned
          before
\xcdmath"if ($e$) $s$".
$v$ is definitely assigned before $s$ iff $v$ is definitely assigned after $e$ when true.
    \item $v$ is definitely assigned after \xcdmath"if ($e$) $s$ else $t$" iff $v$ is definitely assigned after $s$ and $v$ is definitely assigned after $t$.
    \item $v$ is definitely assigned before $e$ iff $v$ is definitely assigned
          before \xcdmath"if ($e$) $s$ else $t$". $v$ is definitely assigned before $s$
          iff $v$ is definitely assigned after $e$ when true. $v$ is
          definitely assigned before $t$ iff $v$ is definitely assigned after
          $e$ when false. 
\end{itemize}

\paragraph{{\tt switch} Statements}
Let $s$ be a \xcd`switch` statement, \xcdmath`switch($e$)$b$`
\begin{itemize}
    \item $v$ is definitely assigned after a switch statement $s$ iff 
          (1) either $b$is empty or $v$ is definitely assigned
              after the last statement of $b$, 
          and 
          (2) $v$ is definitely assigned before every \xcd`break` statement
              that may exit $s$.
    \item $v$ is definitely assigned before $e$ iff $v$ is definitely assigned before $s$.
    \item $v$ is definitely assigned before a statement or local variable
          declaration statement $s$ in $b$ iff either 
          (1) $v$ is definitely assigned after $e$, or 
          (2) $s$ is not labeled by a case or default label and $v$ is
              definitely assigned after the preceding statement.  
\end{itemize}

\paragraph{ {\tt while}  Statements}
Let $w$ be a \xcd`while` statement  \xcdmath"while ($e$) $s$".  

\begin{itemize}    \item $v$ is definitely assigned after $w$ iff $v$ is definitely assigned after $e$ when false and $v$ is definitely
assigned before every \xcd`break` statement that may exit $w$.
    \item $v$ is definitely assigned before $e$ iff $v$ is definitely assigned
          before $w$.
    \item $v$ is definitely assigned before $s$ iff $v$ is definitely assigned after $e$ when true. 
\end{itemize}

\paragraph{ {\tt do} Statements}
Let $w$ be a \xcd`do-while` statement  \xcdmath"do $s$ while ($e$)".  

\begin{itemize}
    \item $v$ is definitely assigned after $w$ iff $v$ is definitely assigned
          after $e$ when false and $v$ is definitely assigned before every
          \xcd`break` statement that may exit $w$.
    \item $v$ is definitely assigned before $s$ iff $v$ is definitely assigned before $w$.
    \item $v$ is definitely assigned before $e$ iff $v$ is definitely assigned after $s$ and $v$ is definitely assigned before every continue statement that may exit $s$.
\end{itemize}

\paragraph{ {\tt for} Statements}

% I hope this is correct and not too cheesy...

Let $f$ be a classic-style \xcd`for` statement, 
\xcdmath"for($I$; $T$; $S$) $B$".  
Let $w$ be the equivalent \xcd`while`-loop.  
$v$ is definitely assigned after $f$ iff $v$ is definitely assigned after $w$,
and similarly for the other predicates. 

%%-- \begin{itemize}
%%--     \item $v$ is definitely assigned after a for statement iff both of the following are true:
%%--           o Either a condition expression is not present or $v$ is definitely assigned after the condition expression when false.
%%--           o $v$ is definitely assigned before every break statement that may exit the for statement. 
%%--     \item $v$ is definitely assigned before the initialization part of the for statement iff $v$ is definitely assigned before the for statement.
%%--     \item $v$ is definitely assigned before the condition part of the for statement iff $v$ is definitely assigned after the initialization part of the for statement.
%%--     \item $v$ is definitely assigned before the contained statement iff either of the following is true:
%%--           o A condition expression is present and $v$ is definitely assigned after the condition expression when true.
%%--           o No condition expression is present and $v$ is definitely assigned after the initialization part of the for statement. 
%%--     \item $v$ is definitely assigned before the incrementation part of the for statement iff $v$ is definitely assigned after the contained statement and $v$ is definitely assigned before every continue statement that may exit the body of the for statement. 
%%-- \end{itemize}
%%-- 16.2.10.1 Initialization Part
%%-- 
%%--     \item If the initialization part of the for statement is a local variable declaration statement, the rules of §16.2.3 apply.
%%--     \item Otherwise, if the initialization part is empty, then $v$ is definitely assigned after the initialization part iff $v$ is definitely assigned before the initialization part.
%%--     \item Otherwise, three rules apply:
%%--           o $v$ is definitely assigned after the initialization part iff $v$ is definitely assigned after the last expression statement in the initialization part.
%%--           o $v$ is definitely assigned before the first expression statement in the initialization part iff $v$ is definitely assigned before the initialization part.
%%--           o $v$ is definitely assigned before an expression statement $E$ other than the first in the initialization part iff $v$ is definitely assigned after the expression statement immediately preceding $E$. 
%%-- 
%%-- 16.2.10.2 Incrementation Part
%%-- 
%%--     \item If the incrementation part of the for statement is empty, then $v$ is definitely assigned after the incrementation part iff $v$ is definitely assigned before the incrementation part.
%%--     \item Otherwise, three rules apply:
%%--           o $v$ is definitely assigned after the incrementation part iff $v$ is definitely assigned after the last expression statement in the incrementation part.
%%--           o $v$ is definitely assigned before the first expression statement in the incrementation part iff $v$ is definitely assigned before the incrementation part.
%%--           o $v$ is definitely assigned before an expression statement $E$ other than the first in the incrementation part iff $v$ is definitely assigned after the expression statement immediately preceding $E$. 
%%-- 

\paragraph{{\tt break}, {\tt continue}, {\tt return}, and {\tt throw} Statements}

$v$ is definitely assigned after any \xcd`break`, \xcd`continue`,
\xcd`return`, or \xcd`throw` statement.  This is because none of these
statements can ever complete normally; they all transfer control to some other
place.  Definite assignment only concerns what happens in normal termination,
and, absent normal termination, all definite assignments hold vacuously.

In a \xcd`return` statement with an expression or a \xcd`throw` statement, $v$
is definitely assigned before the expression iff $v$ is definitely assigned
before the \xcd`return` or \xcd`throw` statement. 

\paragraph{ {\tt try} Statements}

Let $t$ be a \xcd`try` statement.  

\begin{itemize}
    \item $v$ is definitely assigned after $t$ iff either 
          (1) $v$ is definitely assigned after the \xcd`try` block and $v$ is
              definitely assigned after every \xcd`catch` block in $t$, or 
          (2) $t$ has a \xcd`finally` block and $v$ is definitely assigned after the \xcd`finally` block. 
    \item $v$ is definitely assigned before the \xcd`try` block iff $v$ is
          definitely assigned before $t$. 
    \item $v$ is definitely assigned before a \xcd`catch` block iff $v$ is
          definitely assigned before the \xcd`try` statement. 

\item $v$ is definitely assigned before a \xcd`finally` block iff $v$ is
      definitely assigned before \xcd`t`.
\end{itemize}
 
\noo{And we need to deal with X10 statements that are not Java -- including
our \xcd`for`}
