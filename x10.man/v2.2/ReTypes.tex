\section{Type System}


\bard{This is the revised 4.1 -- it goes after the intro, but I'm putting it
here to work on it.}

X10 has several sorts of types. 
In this section, \xcd`S`, \xcd`T`, and 
\xcdmath"T$_i$" range over types.  \xcd`X` ranges over type variables, 
\xcd`M` and \xcdmath"x$_i$" over identifiers, 
\xcd`c` over constraint expressions (\Sref{ConstrainedTypes}), 
and \xcdmath"e$_i$" over expressions.\footnote{The actual grammar, as given in 
\Sref{Grammar}, is slightly more intricate for technical reasons.  
The set of types is the same, however, and this grammar is better for
exposition.} 


\begin{bbgrammar}
Type \: \xcd`T` \\
\xcd`T` \: \xcd`M`  & (1)\\
  \: \xcd`X` & (2) \\
  \: \xcd`M` \xcd"["\xcd`T`$_1$ \xcd`,` $\ldots$ \xcd`,` \xcd`T`$_n$ \xcd`]` & (3)\\
  \: \xcd`T`$_1$  \xcd`.` \xcd`T`$_2$ & (4)\\
  \: \xcd`F` & (5)\\
  \: \xcd`M` \xcd"["\xcd`T`$_1$ \xcd`,` $\ldots$ \xcd`,` \xcd`T`$_n$ \xcd`]` 
       \xcd`(` e$_1$ \xcd`,`$\ldots$\xcd`,` e$_k$ \xcd`)` & (6) \\
  \: \xcd`M` 
       \xcd`(` e$_1$ \xcd`,`$\ldots$\xcd`,` e$_k$ \xcd`)` & (6) \\
\xcd`F` \: \xcd`(` \xcd`x`$_1$ \xcd`:` \xcd`T`$_1$ \xcd`,`$\ldots$\xcd`,`
             \xcd`x`$_n$ \xcd`:` \xcd`T`$_n$ \xcd`) {` c  \xcd`} => ` \xcd`T` \\

  \: \xcd`(` \xcd`x`$_1$ \xcd`:` \xcd`T`$_1$ \xcd`,`$\ldots$\xcd`,`
             \xcd`x`$_n$ \xcd`:` \xcd`T`$_n$ \xcd`) {` c  \xcd`} => ` \xcd`void` 
\end{bbgrammar}

The types mean: 
\begin{enumerate}
\item A type given by an identifier \xcd`M`, like \xcd`Point`, \xcd`Int`, or
      \xcd`int`, can refer to a unit -- a class, struct type, or interface.
      \Sref{nuUnitType}.
      Or, it can refer to a name defined by a \xcd`type` statement,
      \Sref{nuTypedef}; \eg, \xcd`int` is an alias for the type \xcd`Int`. 
\item A type variable \xcd`X` refers to a parameter type of a generic
      (parameterized) type, as described in \Sref{nuParameterizedType}.
\item A type \xcd`M[T,U]` is an instance of a generic type, also described in 
      \Sref{nuParameterizedType}.
\item A type \xcd`T.U` is a qualified type: a unit \xcd`U` appearing inside of
      the unit \xcd`T`, as described in \Sref{nuQualifiedTypes}.
\item A type \xcd`F`, such as \xcd`(x:Int){true}=>Int`, is a function type.
      Its values are functions, \eg, taking integers to integers. They are
      described in \Sref{nuFunctionTypes}.
\item A term \xcd`M[T](e)` is an instance of a parameterized type definition.  
      \eg, \xcd`Array[Int](1)` is the type of one-dimensional arrays of
      integers, and \xcd`Region(1)` is the type of one-dimensional regions of
      points (\Sref{XtenRegions}).  
      This is described in \Sref{nuTypedef}.
\end{enumerate}
