\documentclass[9pt]{sigplanconf}


\conferenceinfo{X10'11,} {June 4, 2011, San Jose, California, USA.}
\CopyrightYear{2011}
%\copyrightdata{}

% Doesn't work in a caption!
%\newcounter{RuleCounter}
%\stepcounter{RuleCounter}
% \arabic{RuleCounter}
\newcommand{\myrule}[2]{\textbf{Rule #1:} #2.}

\usepackage{relsize}
\usepackage{amsmath}
\usepackage{url}


% http://en.wikibooks.org/wiki/LaTeX/Packages/Listings
\usepackage{listings}


\usepackage[ruled]{algorithm} % [plain]
\usepackage[noend]{algorithmic} % [noend]
\renewcommand\algorithmiccomment[1]{// \textit{#1}} %


% For fancy end of line formatting.
\usepackage{microtype}


% For smaller font.
\usepackage{pslatex}


\usepackage{xspace}
\usepackage{yglabels}
\usepackage{yglang}
\usepackage{ygequation}
\usepackage{graphicx}
%\usepackage{epstopdf}

\newcommand{\formalrule}[1]{\mbox{\textsc{\scriptsize #1}}}
\newcommand{\myrule}[1]{\textsc{\codesmaller #1 Rule}}
\newcommand{\umyrule}[1]{\textbf{\underline{\textsc{\codesmaller #1 Rule}}}}


% \small \footnotesize \scriptsize \tiny
% \codesize and \scriptsize seem to do the same thing.
% \newcommand{\code}[1]{\texttt{\textup{\footnotesize #1}}}
% \newcommand{\code}[1]{\texttt{\textup{\codesize #1}}}
\newcommand{\normalcode}[1]{\texttt{\textup{#1}}}
\def\codesmaller{\small}
\newcommand{\myCOMMENT}[1]{\COMMENT{\small #1}}
\newcommand{\code}[1]{\texttt{\textup{\codesmaller #1}}}
% \newcommand{\code}[1]{\ifmmode{\mbox{\smaller\ttfamily{#1}}}
%                       \else{\smaller\ttfamily #1}\fi}
\newcommand{\smallcode}[1]{\texttt{\textup{\scriptsize #1}}}
%\newcommand{\myparagraph}[1]{\noindent\textit{\textbf{#1}}~} %\vspace{-1mm}\paragraph{#1}}

% See: \usepackage{bold-extra} if you want to do \textbf
\newcommand{\keyword}[1]{\code{#1}}

% For new, method invocation, and cast:
\newcommand{\hparen}[1]{\code{(}#1\code{)}}

\newcommand{\hgn}[1]{\lt#1\gt} % type parameters and generic method parameters

\newcommand{\Own}{{\it O}}
\newcommand{\Ifn}[1]{\ensuremath{I(#1)}}
\newcommand{\Ofn}[1]{\ensuremath{O(#1)}}
\newcommand{\Cooker}[1]{\ensuremath{{\kappa}(#1)}}
\newcommand{\Owner}[1]{\ensuremath{{\theta}(#1)}}
\newcommand{\Oprec}[0]{\ensuremath{\preceq_{\theta}}}
\newcommand{\Tprec}[0]{\ensuremath{\preceq^T}}
\newcommand{\TprecNotEqual}[0]{\ensuremath{\prec^T}}
\newcommand{\OprecNotEqual}[0]{\ensuremath{\prec_{\theta}}}
\newcommand{\IfnDelta}[1]{\ensuremath{I_\Delta(#1)}}
\newcommand\Abs[1]{\ensuremath{\left\lvert#1\right\rvert}}
\newcommand{\erase}[1]{\ensuremath{\Abs{#1}}}

\newcommand{\CookerH}[1]{\ensuremath{{\kappa}_H(#1)}}
\newcommand{\IfnH}[1]{\ensuremath{I_H(#1)}}
\newcommand{\OwnerH}[1]{\ensuremath{{\theta}_H(#1)}}
\newcommand{\Gdash}[0]{\ensuremath{\Gamma \vdash }}
\newcommand{\reducesto}[0]{\rightsquigarrow}
\newcommand{\reduce}[0]{\rightsquigarrow}

\usepackage{color}
\definecolor{light}{gray}{.75}


\newcommand{\todo}[1]{\textbf{[[#1]]}}
%% To disable, just uncomment this line:
%\renewcommand{\todo}[1]{\relax}

%% Additional todo commands:
\newcommand{\TODO}[1]{\todo{TODO: #1}}

\newcommand\xX[1]{$\textsuperscript{\textit{\text{#1}}}$}


\newcommand{\ol}[1]{\overline{#1}}
\newcommand{\nounderline}[1]{{#1}}


%% Commands used to typeset the FIGJ type system.
\newcommand{\typerule}[2]{
\begin{array}{c}
  #1 \\
\hline
  #2
\end{array}}


%% Commands used to typeset the FOIGJ type system.
\newcommand{\inside}{\prec}
\newcommand{\visible}{{\it visible}}
\newcommand{\placeholderowners}{{\it placeholderowners}}
\newcommand{\nullexpression}{{\tt null}}
\newcommand{\errorexpression}{{\tt error}}
\newcommand{\locations}{{\it locations}} %% \mathop{\mathit{locations}}}
\newcommand{\xo}{{\tt X^O}}
\newcommand{\no}{{\tt N^O}}
\newcommand{\co}{{\tt C^O}}
\newcommand{\I}{\it I}


\newcommand\mynewcommand[2]{\newcommand{#1}{#2\xspace}}


\mynewcommand{\hI}{\code{I}} % iparam

% In the syntax: \hI or ReadOnly or Mutable or Immut
\mynewcommand{\hJ}{\code{J}}
\mynewcommand{\hO}{\code{O}}
\mynewcommand{\ho}{\code{o}}
\mynewcommand{\hnull}{\code{null}}
\mynewcommand{\htrue}{\code{true}}
\mynewcommand{\hfalse}{\code{false}}

\mynewcommand{\hX}{\code{X}} % vars
\mynewcommand{\hY}{\code{Y}} % vars
\mynewcommand{\hB}{\code{B}} % class
\mynewcommand{\hC}{\code{C}} % class
\mynewcommand{\hD}{\code{D}} % class
\mynewcommand{\hL}{\code{L}} % class decl
\mynewcommand{\hM}{\code{M}} % Method decl
\mynewcommand{\hm}{\code{m}} % method
\mynewcommand{\he}{\code{e}} % expression
\mynewcommand{\hl}{\code{l}} % location in the store
\mynewcommand{\hx}{\code{x}} % method parameter
\mynewcommand{\hf}{\code{f}} % field name
\mynewcommand{\hF}{\code{F}} % field
\mynewcommand{\hH}{\code{H}} % Heap
\mynewcommand{\hS}{\code{S}}
\mynewcommand{\hsub}{\code{/}} % substitute (reduction rules)


\mynewcommand{\hp}{\code{p}} % path: either l or x
\mynewcommand{\hs}{\code{s}}
\mynewcommand{\hT}{\code{T}}

\mynewcommand{\hand}{\code{~and~}}
\mynewcommand{\hor}{\code{~or~}}
\mynewcommand{\hthis}{\code{this}} % this
\mynewcommand{\super}{\code{super}} % this
\mynewcommand{\hsuper}{\code{super}} % this
\mynewcommand{\hclass}{\code{class}}
\mynewcommand{\hreturn}{\code{return}}
\mynewcommand{\hhnew}{\code{new}}
\newcommand{\myinit}[2]{<#1,#2>}
\mynewcommand{\mycooked}{\ \myinit{\emptyset}{\emptyset}}
\newcommand{\hnew}[1]{\code{new}~#1}
\newcommand{\hlet}[3]{\code{let}~#1~=~#2~\code{in}~#3}

\newcommand{\lt}{\code{<}}%{\mathop{\textrm{\tt <}}}
\newcommand{\gt}{\code{>}}%{\mathop{\textrm{\tt >}}}

\mynewcommand{\this}{\keyword{this}}
\mynewcommand{\ctor}{\keyword{ctor}}
\mynewcommand{\Object}{\code{Object}}
\mynewcommand{\const}{\keyword{const}} %C++ keyword
\mynewcommand{\mutable}{\keyword{mutable}} %C++ keyword
\mynewcommand{\romaybe}{\keyword{romaybe}} %Javari keyword

%% Define the behaviour of the theorem package.
%% Use http://math.ucsd.edu/~jeggers/latex/amsthdoc.pdf for reference.

\newtheorem{theorem}{Theorem}[section]
\newtheorem{definition}[theorem]{Definition}
\newtheorem{lemma}[theorem]{Lemma}
\newtheorem{corollary}[theorem]{Corollary}
\newtheorem{fact}[theorem]{Fact}
\newtheorem{example}[theorem]{Example}
\newtheorem{remark}[theorem]{Remark}


\mynewcommand{\IP}{\code{I}}   % formal type parameter
\mynewcommand{\JP}{\code{J}}   % formal type parameter (for soundness proofs)


\mynewcommand{\Iparam}{Immutability parameter}
\mynewcommand{\iparam}{immutability parameter}
\mynewcommand{\iparams}{immutability parameters}
\mynewcommand{\Iparams}{Immutability parameters}
\mynewcommand{\Iarg}{Immutability argument}
\mynewcommand{\iarg}{immutability argument}
\mynewcommand{\iargs}{immutability arguments}
\mynewcommand{\Iargs}{Immutability arguments}
\mynewcommand{\ReadOnly}{\code{ReadOnly}}
\mynewcommand{\WriteOnly}{\code{WriteOnly}}
\mynewcommand{\None}{\code{None}}
\mynewcommand{\Mutable}{\code{Mutable}}
\mynewcommand{\Immut}{\code{Immut}}
\mynewcommand{\Raw}{\code{Raw}}


\mynewcommand{\This}{\code{This}}
\mynewcommand{\World}{\code{World}}


% Our annotations
\mynewcommand{\OMutable}{\code{@OMutable}}
\mynewcommand{\OI}{\code{@OI}}


\mynewcommand{\InVariantAnnot}{\code{@InVariant}}


\newcommand{\func}[1]{\text{\textnormal{\textit{\codesmaller #1}}}}


\mynewcommand{\st}{\ensuremath{\mathrel{{\leq}}}} %{\mathop{\textrm{\tt <:}}}
\mynewcommand{\notst}{\mathrel{\st\hspace{-1.5ex}\rule[-.25em]{.4pt}{1em}~}}
\mynewcommand{\tl}{\ensuremath{\triangleleft}}
\mynewcommand{\gap}{~ ~ ~ ~ ~ ~}


\newcommand{\RULE}[1]{\textsc{\scriptsize{}#1}} %\RULEhape\scriptsize}


\mynewcommand{\DA}{\texttt{DA}}
\mynewcommand{\ok}{\texttt{OK}}
\mynewcommand{\OK}{\texttt{OK}}
\mynewcommand{\IN}{\texttt{IN}}
\mynewcommand{\subterm}{\func{subterm}}
\mynewcommand{\TP}{\func{TP}} % function that returns type parameters in a type
\mynewcommand{\CT}{\func{CT}} % class table
\mynewcommand{\mtype}{\func{mtype}}
\mynewcommand{\mmodifier}{\func{mmodifier}}
\mynewcommand{\fmodifier}{\func{fmodifier}}
\mynewcommand{\mbody}{\func{mbody}}
\mynewcommand{\ftype}{\func{ftype}}
\mynewcommand{\fields}{\func{fields}}
\DeclareMathOperator{\dom}{dom}
\mynewcommand{\reductionVal}{\func{reductionVal}} % val assigned at most once


\mynewcommand{\finishG}{\func{finish}}
\mynewcommand{\asyncG}{\func{async}}


\mynewcommand\xth{\xX{th}}
\mynewcommand\xrd{\xX{rd}}
\mynewcommand\xnd{\xX{nd}}
\mynewcommand\xst{\xX{st}}
\mynewcommand\ith{$i$\xth}
\mynewcommand\jth{$j$\xth}


%\mynewcommand{\emptyline}{\vspace{\baselineskip}}
\mynewcommand{\myindent}{~~}


% Add line between figure and text
\makeatletter
\def\topfigrule{\kern3\p@ \hrule \kern -3.4\p@} % the \hrule is .4pt high
\def\botfigrule{\kern-3\p@ \hrule \kern 2.6\p@} % the \hrule is .4pt high
\def\dblfigrule{\kern3\p@ \hrule \kern -3.4\p@} % the \hrule is .4pt high
\makeatother

\setlength{\textfloatsep}{.75\textfloatsep}


% Remove line between figure and its caption.  (The line is prettier, and
% it also saves a couple column-inches.)
\makeatletter
%\@setflag \@caprule = \@false
\makeatother


% http://www.tex.ac.uk/cgi-bin/texfaq2html?label=bold-extras
\usepackage{bold-extra}


% Left and right curly braces in tt font
\newcommand{\ttlcb}{\texttt{\char "7B}}
\newcommand{\ttrcb}{\texttt{\char "7D}}
\newcommand{\lb}{\ttlcb}
\newcommand{\rb}{\ttrcb}


\setlength{\leftmargini}{.75\leftmargini}


\begin{document}


\lstset{language=java,basicstyle=\ttfamily\small}


\title{Object Initialization in X10}


\authorinfo{Yoav Zibin \and Igor Peshansky \and David Cunningham \and Vijay Saraswat}
           {IBM research in TJ Watson}
           {yzibin$|$igorp$|$dcunnin$|$vsaraswa@us.ibm.com}


\maketitle


\begin{abstract}
%\title{Sequential semantics for concurrent programs}


Imperative memory locations are famously indeterminate under
concurrent read/write operations. Further, to support reliable
communication most concurrent programming languages support
synchronization mechanisms that enable a thread to wait until some
appropriate condition is true, thus leading to the possibility of
deadlock. Sequential programs are fundamentally determinate and
deadlock-free; thus it is very difficult to systematically write
concurrent programs with sequential semantics, even if the program is
in fact a parallelization of a sequential program. 

In this paper we develop a fragment of the programming language
\Xten{} in which all programs have sequential semantics.  \Xten{} is
based on a very powerful imperative calculus supporting fine-grained
asynchronous concurrency, multi-place execution, messaging and clocked
synchronization. We introduce two new abstractions -- {\em
  accumulators} and {\em clocked values} -- with lightweight compiler
and run-time support. Accumulators of type \code{T} permit multiple
concurrent writes, these are reduced into a single value by a
user-specified reduction operator. Clocked values of type \code{T}
operate on two values of \code{T} (the {\em current} and the {\em
  next}). Read operations are performed on the current value and write
operations on the next. Such values are implicitly associated with a
clock and the current and next values are switched (determinately) on
quiescence of the clock. Clocked values capture the common ``double
buffering'' or ``red/black'' concurrency idiom.

We introduce
the notion 

In this paper we present some lightweight techniques that enable a
large class of concurrent programs to be written in such a way that
their semantics 

The former can lead to run-time overheads and the latter to brittle
programs since it is very difficult to smoothly extend the rich type
systems underlying modern imperative languages (supporting objects,
inheritance, subtyping, genericity etc) to support determinacy. The
fundamental problem is the possibility of aliasing in arbitrary
unstructured heaps. This makes it very difficult to get a static
handle on concurrent access to a shared location.

In this paper we show that a middle path can be successful for a
language based on structured concurrency, such as \Xten.  We introduce
two new abstractions -- 


We show that these abstractions (by design) determinate and
deadlock-free in any usage, though executions may throw exceptions
under certain circumstances. We show that these data-structures are
very natural to use and successfully capture many common patterns of
expression that are semantically determinate (e.g. histograms,
all-to-all reductions, stencil computations etc). We show that there
are simple statically-checkable rules that can establish for many
common idioms that concurrency-related exceptions will not be thrown
at run-time, and that some potentially costly synchronization checks
can be avoided.

A key technical innovation is the introduction of an {\em implicit
  ownership domain} for objects. This provides a way around the
unstructured heap by permitting only the current activity and its
spawned asyncs to access the objects. Now the block structure of
\code{finish}, \code{async}, \code{at} and the clock construct of
\Xten{} can be used to establish determinacy in a local way,
independent of the context in which the code is being used. 


\end{abstract}

\category{D.3.3}{Programming Languages}{Language Constructs and Features}
\category{D.1.5}{Programming Techniques}{Object-oriented Programming}

\terms
Asynchronous, Initialization, Types, X10

% Keywords are not required in the paper itself - only in the submission system's meta data.
% \keywords
% Immutability, Ownership, Java


\Section[introduction]{Introduction}
A serial schedule for a parallel program is one which always executes
the first enabled step in program order. A {\em safe} parallel program
is one that can be executed with a serial schedule $S$ and for which
every schedule produces the same result as $S$.  Such a program is
semantically a sequential program, hence it is scheduler-determinate
and deadlock-free.

A {\em safe programming language} is an imperative parallel
programming language in which every legal program is a safe
program. Programmers can write code in such a language secure in the
knowledge that they will not encounter a large class of parallel
programming problems. Such a language is particularly useful for
parallelizing sequential (imperative) programs. In such cases ({\em
  contra} reactive programming) the desired application semantics are
sequential, and parallelism is needed purely for efficient
implementation.

The key property of a safe programming language is that the {\em same}
program can be developed and debugged as a sequential program and then
safely run in parallel. Parallel execution is guaranteed to effect
only performance, not correctness. Safety is a very strong property.

Figure~\ref{fig:1} shows  the famous ``parallel Or'' program of Plotkin 
 (in X10 syntax, \cite{x10}). This program can be executed with a
depth-first schedule, is partially determinate and deadlock-free, but {\em not}
safe. The result of running the sequential schedule is not the same as
the result that can be obtained with other schedules. Specifically
\code{parallelOr(()=> CONT, ()=>TRUE)} will diverge (exhibit an
infinite exection sequence) under the depth-first schedule, but will
return \code{true} under any fair schedule that permits the second
async to progress.

\begin{figure}
  \begin{lstlisting}
static val CONT=1, TRUE=2, FALSE=3;
def run(done:Cell[Boolean], a:()=>Int) {
 var aa:Int=a();
 var cont:Boolean=true;         
   for (; aa==CONT && cont;aa=a()) {
      atomic cont = !done();  
   }
   if (aa==TRUE)
     atomic done()=true;
}
def parallelOr(a:()=>Int, b:()=>Int):Boolean {
  val done=new Cell[Boolean](false);
  finish {
    async run(done, a);
    async run(done, b);
  }
  return done();
}
  \end{lstlisting}
  \caption{A program that is not sequential}\label{fig:1}
\end{figure}
In this paper we establish that a very rich fragment of X10 is safe.
The fragment supports multi-place, fine-grained concurrency
over an arbitrary heap, clocked computations, concurrent accumulators,
and {\em clocked types} that safely capture the ``red black'' idiom
for iterative computations. It also uses a very lightweight effects
mechanism to reason about disjointness of acccess. We show that a
large variety of concurrency and communication idioms can be naturally
expressed in Safe X10.

Determinism in parallel programming is a very active area of research.
Guava \cite{guava} introduces restrictions on shared memory Java
programs that ensure no data-races primarily by distinguishing
monitors (all access is synchronized) from values (immutable) and
objects (private to a thread). However Guava is not safe since Guava
programs may use \code{wait}/\code{notify} for arbitrary concurrent
signalling and hence may not executable with a sequential
schedule. The Revisions programming model \cite{Revisions} guarantees
determinism by isolating asynchronous tasks but merging their writes
determinately. However, the model explicitly does not require that
a sequential schedule be valid (c.f. Figure~1 in \cite{Revisions}).

DPJ develops the ``determinacy-by-default'' slogan using a static
type-and-effects system to establish commutativity of concurrent
actions.  The deterministic fragment of DPJ is safe according to the
definition above. Safe X10 offers a much richer concurrency model
which guarantees the safety of common idioms such as accumulators and
cyclic tasks (clocks) without relying on effects annotations. The
lightweight effects mechanism in X10 can be extended to support a much
richer effects framework (along the lines of DPJ) using X10's
constrained type system.  We leave this as future work.

The SafeJava language \cite{SafeJava} is unfortunately not safe
according to our definition, even though it guarantees determinacy and
deadlock freedom, using ownership types, unique pointers and partially
ordered lock levels. Again, a sequential scheduler is not admissible
for the model.

Some data-flow synchronization based languages and frameworks (e.g.{}
Kahn style process networks \cite{kahn,kahn-mcqueen}, concurrent
constraint programming \cite{ccp}, \cite{SHIM}) are guaranteed
determinate but not safe according to our definition since they do not
permit sequential schedules. Indeed they permit the possibility of
deadlock. (The notion of safety is also not quite relevant since these
frameworks do not support shared mutable variables.)

Determinizing run-times support coarse-grained fork-join concurrency
by maintaining a different copy of memory for each activity and
merging them determinately at finish points (\cite{grace},
\cite{core-det}, \cite{dmp}, \cite{kendo},\cite{determinator}). Safe
X10 can run on such systems in principle, but does not require them.
To execute Safe X10, such systems need to support fine-grained
asynchrony (with some form of work-stealing or fork-joining
scheduler), clocks and accumulators.




%% Anything from Blelloch?

Desiderata:

\begin{itemize}
\item Data-structures should by design by dynamically determinate and
   deadlock-free.
\item They should be first-class -- they can be stored in
  data-structures/read from them, passed as arguments to
  procedures/returned etc with no restrictions. 
\item Usable -- common idioms should be naturally and elegantly expressible.
\item Additional static type-checking can provide extra guarantees
  (e.g.{} no concurrency related run-time exceptions) that may aid
  efficient implementation.
\end{itemize}

We discuss three examples

  * clocks
  * accumulators
  * clocked types

Challenge
  Arbitrary nature of object graphs.


{\em 
\begin{enumerate}
\item Use activity registration as a mechanism to tame object graphs.
\item Focus on structured concurrency. Using scoping and block-structure
    to delimit regions of code that may execute in parallel and affect
    the data structure.

\item Accumulation can be defined safely by delaying. However, the delay
    operation is guaranteed to be deadlock-free.

\item Clocked types support phased computation, another common idiom
    particularly for stencil computations.
\end{enumerate}
}

Key contributions:
{\em 
\begin{enumerate}
\item Identification of determinate, deadlock-free data-structures. 
\item Discussion of design alternatives which points out the
  difficulty of integrating these ideas in a modern OO language.
\item Discussion of various idioms expressible using these data-structures.
\item Proof of determinacy and deadlock-freedom in an abstract version
  of the language.
\end{enumerate}
These constructs are implemented in \Xten, available as open source from
SVN head and will be in the next release of \Xten.
}


Semantics and theorems for an abstract version of the language.


\subsection{Related Work}

\cite{Steele:1989:MAP:96709.96731} introduced the idea of 


\Section[designs]{Object Initialization Designs}


\subsection{Default values design}

Java first initializes fields with either $0$, \texttt{false}, or \texttt{null}
(depending on the field type) and then running the constructor to initialize
the fields according to the programmer's wishes.  If every X10 type had a
default value that was statically known, then Java's object initialization
scheme could be used in X10.  This would have the advantage of familiarity for
Java programmers that are learning X10.  The disadvantages are that that it is
nonintuitive that final fields can be observed to change value, and that it is
prone to undetectable errors where the field is read before initialization.

Unfortunately it is hard to reconcile the notion of a default value with X10's
type system, because a programmer can define a type which does not contain the
default value.  In the X10 type system, one can define a type with no values at
all, by using a constraint that yields contradiction.

This could be addressed by extending the X10 type to require that the
programmer define a new constant value for any type that has been constrained
enough that the original default value is no longer a member of the type.  This
means every field can be initialized to the value defined in its type.  The
disadvantage of this is that the type system becomes more complex and more type
annotations are required.  We decided that this, in combination with the
disadvantages given above, was too problematic to justify the advantages of
Java-style object initialization.

\subsection{Proto Design}

If we want to allow some of the programs that the Hardhat design rejects, such
as immutable cycles in the object graph, but we do not want to burden the type
system with default value annotations, then one solution is to allow
\texttt{this} to escape in certain cases while still preventing reads from
uninitialized fields.  This can be achieved by annotating reference types with
a keyword \texttt{proto} to indicate that the referenced object is partially
constructed.  Reads of fields where the target object has \texttt{proto} type
are not allowed because a partially constructed object may not yet have
initialized its fields.  The advantage of this approach is that it allows a set
of partially constructed objects to establish themselves as a cycle of field
references.  The disadvantage is it requires an additional type annotations,
although this annotation is only required if one wants to create immutable
cyclic heap structures.  Also note that there are no additional space or
runtime overheads since these extra type system mechanisms are for static
checking only.

An example of an immutable cycle of two nodes is given
in fig.\ref{Figure:Cyclic}.  A more practical but less concise example would be
an immutable doubly-linked list.  Let us assume that we would like to optimize
away any null pointer checks, so we constrain all references to exclude the
null value.

\begin{figure}
\begin{lstlisting}
class C {
    public val next : C {self!=null};
    private def this (n:proto C{self!=null}) {
        // Console.OUT.println(n.next);
        // n.f();
        // s(n);
        this.next = n;
    }
    public def this () {
        // Console.OUT.println(this.next);
        // f();
        // s(this);
        this.next = new C(this);
    }
    public def f() {
        Console.OUT.println(this.next);
    }
    public static def s(C this_) {
        this_.f();
    }
}
val c:C{self!=null} = new C();
val c2:C{self!=null} = c.next;
\end{lstlisting}
\caption{An immutable cycle of heap references, using \texttt{proto}.}
\label{Figure:Cyclic}
\end{figure}

The commented out lines indicate code that would be rejected by the type
system.  In the public constructor, \texttt{this} is a pointer to a partially
constructed object.  If the type of \texttt{this} were to be explicit, it would
be \texttt{proto C\{self!=null\}}.  The \texttt{proto} element of the type
forbids any field reads, and permits writes to immutable fields.  It also
prevents the reference being leaked (e.g. into \texttt{f()}), except into
variables where the \texttt{proto} type is also present and therefore where
there is protection from uninitialized field reads.

The private constructor's \texttt{n} parameter takes a \texttt{proto} pointer
to the original \texttt{C} instance.  It is very limited in what it can do,
e.g. it cannot read \texttt{n.next}, but it can initialize its \texttt{next}
field with the passed-in value.  When the public constructor returns, both
objects are fully constructed with all fields initialized.  Thus, the type of
the variable \texttt{c} does not have a proto annotation and the field read
\texttt{c.next} is allowed.

If a type has the \texttt{proto} keyword, then the instance is definitely
partially constructed.  Likewise, the absence of \texttt{proto} means the
instance is definitely fully constructed.  Thus, there is no subtype
relationship between \texttt{proto C} and \texttt{C}.  It therefore makes no
sense to allow casting between the two types, and one may not extend a proto
type.  The only way to get an objedct of \texttt{proto} type is via the
\texttt{this} keyword in a constructor.

We also do not allow fields to have \texttt{proto} type.  This is because the
referenced object will eventually be fully-constructed and then there would be
a variable of \texttt{proto} type pointing to a fully constructed instance.
This admits the possibility of someone assigning a partially constructed object
to a field of the fully constructed object, just as was done in the private
constructor in fig.~\ref{Figure:Cyclic}.  Then, one could accidently read an
uninitialized field from the partially constructed object by going through the
fully constructed objcet.  Disallowing \texttt{proto} in fields avoids this
problem.  The same problem does not exist with local variables, as due to
lexical scoping, the variable will go out of scope before the constructor
returns and the object becomes fully constructed.

If one examines the state of the heap as this example executes, there can be
seen to be a subgraph of 2 partially-constructed objects, which is completely
isolated except for references from the stack of the thread which is executing
the constructors.  In general, we need for partially constructed objects to be
isolated from the rest of the heap, and this is enforced by disallowing
\texttt{proto} on fields.  The stack reference that causes the subgraph to
remain alive, is safe because it is annotated with \texttt{proto}.

We are not aware of any utility in throwing or catching \texttt{proto} types so
we avoid issues relating to partially constructed objects escaping via the
exception mechanism by simply disallowing the throwing and catching of
\texttt{proto} exceptions.

There would be an issue calling other instance methods on \texttt{this} from a
constructor, because the type of \texttt{this} in those methods would need to
be \texttt{proto} since the target is still partially constructed.  We support
this by allowing the \texttt{proto} keyword to also be used on a method as an
effect annotation, i.e. it must be preserved by inheritance.  Such methods can
only becalled on partially constructed objects, and the type of \texttt{this}
subjects them to the same restrictions as in constructor bodies.

While we believe this type system is correct and usable for writing real
programs in the X10 language, we had to decide whether the additional type
system complexity and annotations were a reasonable price to pay for the
additional expressiveness (i.e. the ability to construct immutable heap
cycles).  We ultimately decided that immutable heap cycles are too rare in
practice to justify including these extra mechanisms in the language.


\Section[implementation]{Hardhat implementation}
implementation design, overheads, some measurements, etc.

outlines our implementation within the X10 compiler using the polyglot framework,
    the compilation time overhead of checking these initialization rules,
    and the annotation overhead in our X10 code base.


Due to page limitation, we mainly focused on the formal effect system for POPL,
but we can easily add an empirical evaluation section that describes some test cases (where minor code refactoring was needed) and shows the annotation burden.
X10 has only two possible method annotations: @NonEscaping, @NoThisAccess.
Methods transitively called from a constructor are implicitly non-escaping (but the compiler issues a warning that they should be marked as @NonEscaping).
SPECjbb and M3R are closed-source whereas the rest is open-source and publicly available at x10-lang.org

------------------------------------------------------------------------------
Programs:           XRX SPECjbb     M3R UTS Other
# of lines          27153   14603       71682   2765    155345
# of files          257 63      294 14  2283
# of constructors       276 267     401 23  1297
# of methods            2216    2475        2831    124 8273
# of non-escaping methods   8   38      34  3   83
# of @NonEscaping       7   7       13  1   62
# of @NoThisAccess      1   0       1   0   12
------------------------------------------------------------------------------
XRX: X10 Runtime (and libraries)
SPECjbb: SPECjbb from 2005 converted to X10
M3R: Map-reduce in X10
UTS: Global load balancing
Other: Programmer guide examples, test suite, issues, samples
------------------------------------------------------------------------------

As can be seen, the annotations burden is minor.

Asynchronous initialization was not used in our applications because they pre-date this feature.
(It is used in our examples and tests 50 times.)
However, it is a useful pattern, especially for local variables.
More importantly, the analysis prevents bugs such as:
val res:Int;
finish {
  async {
    res = doCalculation();
  }
  // WRONG to use res here
}
// OK to use res here

Here are two examples for the use of annotations:
1) In Any.x10 we have:
@NoThisAccess def typeName():String
Method typeName is overridden in subclasses to return a constant string (all structs automatically override this method).
This annotation allows typeName() to be called even during construction.
2) In HashMap.x10, after we added the strict initializations rules, we had to refactor put and rehash methods into:
public def put(k: K, v: V) = putInternal(k,v);
@NonEscaping protected final def putInternal(k: K, v: V) {...}
(Similarly, we have rehash() and rehashInternal())
The reason is that putInternal is called from the deserialization constructor:
def this(x:SerialData) { ... putInternal(...) ... }
And we still want subclasses to be able to override the "put" method.


\Section[case-study]{Case Study}
\input{case-study}

\Section[related-work]{Related Work}
\label{sec:related}

\subsection{Programming Models} 
Determinism in parallel programming is a very active area of research.
Guava \cite{guava} introduces restrictions on shared memory Java
programs that ensure no data-races primarily by distinguishing
monitors (all access is synchronized) from values (immutable) and
objects (private to a thread). However Guava is not safe since Guava
programs may use \code{wait}/\code{notify} for arbitrary concurrent
signalling and hence may not executable with a sequential
schedule. The Revisions programming model \cite{Revisions} guarantees
determinism by isolating asynchronous tasks but merging their writes
determinately. However, the model explicitly does not require that
a sequential schedule be valid (c.f. Figure~1 in \cite{Revisions}).

%\cite{Steele:1989:MAP:96709.96731} introduced the idea that

DPJ develops the ``determinacy-by-default'' slogan using a static
type-and-effects system to establish commutativity of concurrent
actions.  The deterministic fragment of DPJ is safe according to the
definition above. Safe X10 offers a much richer concurrency model
which guarantees the safety of common idioms such as accumulators and
cyclic tasks (clocks) without relying on effects annotations. The
lightweight effects mechanism in X10 can be extended to support a much
richer effects framework (along the lines of DPJ) using X10's
constrained type system.  We leave this as future work.

The SafeJava language \cite{SafeJava} is unfortunately not safe
according to our definition, even though it guarantees determinacy and
deadlock freedom, using ownership types, unique pointers and partially
ordered lock levels. Again, a sequential scheduler is not admissible
for the model.

Some data-flow synchronization based languages and frameworks (e.g.{}
Kahn style process networks \cite{kahn,kahn-mcqueen}, concurrent
constraint programming \cite{ccp}, \cite{SHIM}) are guaranteed
determinate but not safe according to our definition since they do not
permit sequential schedules. Indeed they permit the possibility of
deadlock. (The notion of safety is also not quite relevant since these
frameworks do not support shared mutable variables.)


Synchronous programming languages like Esterel are completely
deterministic. An Esterel program executes in clock steps and the
outputs are conceptually synchronous with its inputs.  It is a finite
state language that is easy to verify formally. An Esterel program is
susceptible to causalities. Causalities are similar to deadlocks, but
can be easily detected at compile-time.  The problem with synchronous
models is that they do not perform well. To out knowledge, most
Esterel compilers generate sequential code and there are hardly any
compilers that generate concurrent code off Esterel.

 
SHIM~\cite{edwards2005shim2,tardieu2006scheduling-independent} is also
a deterministic concurrent programming language, but the improper use
of its constructs leads to problems such as deadlocks i.e., a SHIM program
may be susceptible to deadlocks. Any program written in our model is always
deadlock-free. Secondly, SHIM allows only a single task to write at any phase;
we allow multiply writes.

 
Apart from SHIM, there are  a few programming models and languages 
that provide explicit determinism. StreamIt~\cite{thies2001streamit}, for 
example is a synchronous dataflow language that provides determinism. It 
has simple static verification techniques for deadlock and buffer-overflow. 
However, StreamIt is a strict subset of SHIM and StreamIt's design 
limits it to a small class of streaming applications. 
 


In contrast, 
Cilk~\cite{blumofe1995cilk} is a non-deterministic language that it covers a larger
class of applications. It is C based
and the programmer must explicitly ask for parallelism using 
the \emph{spawn} and the \emph{sync} constructs. 
Cilk is definitely more
expressive than $D^2C$.
However, Cilk allows data races. 
%\figref{non-det}, for example,
%is a non-deterministic concurrent program in Cilk. 
Explicit techniques~\cite{cheng1998detecting} are
required for checking data races in Cilk programs.  




\subsection{Determinizing Tools} 
Determinizing run-times support coarse-grained fork-join concurrency
by maintaining a different copy of memory for each activity and
merging them determinately at finish points (\cite{grace},
\cite{core-det}, \cite{dmp}, \cite{kendo},\cite{determinator}). Safe
X10 can run on such systems in principle, but does not require them.
To execute Safe X10, such systems need to support fine-grained
asynchrony (with some form of work-stealing or fork-joining
scheduler), clocks and accumulators.

Kendo is a purely software system that deterministically multi-threads
concurrent applications.  Kendo~\cite{olszewski2009kendo} ensures a
deterministic order of all lock acquisitions for a given program
input.

Kendo comes with three shortcomings. It operates completely at runtime,
and there is considerable performance penalty. Secondly, if
we have the sequence \emph{lock(A); lock (B)} in one thread and
\emph{lock(B); lock(A)} in another thread, a deterministic ordering of
locks may still deadlock. Thirdly, the tool operates only when
shared data is protected by locks.

Software Transactional Memory (STM)~\cite{shavit1995software}
  is an alternative to locks: a thread completes modifications to 
shared memory without regard for what other threads might be doing. At the end of the transaction,
it validates and commits if the validation was successful, otherwise it rolls back and re-executes
the transaction. STM mechanisms avoid races but do not solve the non-determinism problem.

Berger's Grace\cite{berger2009grace} is a run-time tool
that is based on STM. 
If there is a conflict during commit, the threads are committed in
a particular sequential order (determined by the order
The problem with Grace is that it incurs a lot of run-time
overhead. This dissertation  partially solves this overhead problem
by addressing the issue at compile-time and
thereby reducing a considerable amount of run-time overhead.

Like Grace, Determinator\cite{aviram2010efficient} is another tool
that allows parallel processes to execute as long as they do not share 
resources. If they do share resources and the accesses are unsafe, then
the operating throws an exception (a page fault). 

Cored-Det~\cite{bergan2010coreDet}, based on DMP~\cite{devietti2009dmp} 
uses a deterministic token that is passed
among all threads.  A thread to modify a shared variable must first
wait for the token and for all threads to block on that
token. DMP is hardware based. 
Although, deadlocks may be avoided, we believe this setting is
non-distributed because it forces all threads to synchronize and
therefore leads to a considerable performance penalty. In the $D^2C$ 
setting, only threads that share a particular channel must synchronize
on that channel; other threads can run independently.

 Deterministic replay systems~\cite{choi1998deterministic,altekar2009odr} facilitate debugging of concurrent programs to produce
repeatable behavior. They are based on record/replay systems. The system
replays a specific behavior (such as thread interleaving) of a concurrent
program based on records. The primary purpose of replay systems 
is debugging; they do not guarantee determinism. 
They incur a high runtime overhead and are input dependent.
For every new input, a new set of records is generally maintained.

Like replay systems, Burmin and Sen~\cite{Burnim2009asserting} provide a framework for
checking determinism for multi-threaded programs. Their tool does not
introduce deadlocks, but their tool does not guarantee determinism
because it is merely a testing tool that checks the execution trace
with previously executed traces to see if the values match. Our
goal is to guarantee determinism at compile time -- given a program,
it will generate the same output for a given input.



\subsection{Type Systems and Verifiers} 
 
Finally, type and effect systems like DPJ~\cite{bocchino2009type} 
 have been designed for deterministic parallel programming to see if
memory locations overlap. Our technique is more explicit. 
In general, type systems require the programmer to manually annotate the program. Our model can also be implemented using annotations in existing
programming languages - we in fact annotated the X10 programming language.

Martin Vechev's tool \cite{vechev2011automatic}
finds determinacy bugs in loops that run parallel bodies. It analyzes
array references and indices to ensure that there are no read-write and 



\Section[conclusion]{Conclusion}
We show that many determinate concurrent programs can be written in
\Xten{} using determinate, deadlock-free constructs, so that they are
determinate by design.



\bibliographystyle{plainnat}
\bibliography{x10-init}

\end{document}
