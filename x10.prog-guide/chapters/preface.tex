\chapter*{Preface}
\section*{Who This Book Is For}
\Xten{} is an experimental language designed with modern programming language
concepts and features to support ``scale out'' and ``scale up'' concurrency:
larger and larger numbers of more and more powerful nodes that have to be used
together effectively. This text introduces \Xten{} to programmers who have some
experience with an object-oriented language.  To this end, we'll talk about Java and C++
from time to time, as a way to explain concepts that these languages share with
\Xten{}.   So, a background in Java or C++ will help in reading this book; failing that, 
familiarity with object-oriented ``scripting'' languages like JavaScript, Python, Ruby, or
Smalltalk will also give much of the required background.

Those familiar with C, but not C++, may find this book useful,
but some minimal knowledge of object-oriented programming really is a must---\eg{}
classes, objects, class methods versus instance methods, and so on, but nothing deep. 

\section*{The Scope Of The Book}
Just as an MPI programmer can go a long way with only a
half-dozen or so of the more than one hundred MPI functions, 
the new \Xten{} programmer can get a lot of mileage from a carefully
chosen subset of  \Xten{}  constructs.  Aiming for a gentle introduction,
we stick to the most fundamental language constructs, that cover the most
common programming tasks.  The final chapter surveys what we don't cover here 
in detail.

Here is a quick outline of what we are going to do.
\begin{quote}
The first chapter samples a  few simple \Xten{} programs so you can get a feeling for the
language.   It introduces concurrency, both local and distributed.  The
chapter emphasizes the basic constructs, not performance, except to the
extent that  it comes up as a reason for choosing one construct over another.
 
The next several chapters flesh out the critical parts of the language:
types, expressions, control flow, and so on.  

\ignore{
In the remainder of the tuorial, we present some more substantial
examples so that you can get a feeling for how  \Xten{} plays in industrial-strength
applications, and performance is a critical part  of that discussion. Most of these
examples are drawn from Michael J. Quinn's
text ``\emph{Parallel Programming in C with  MPI and OpenMP}'', published by
McGraw-Hill in 2004, which is  a  very readable basic introduction to parallel
computing. 
}
\end{quote}

\ignore{
Although we try to be reasonably self-contained, we rely on Quinn's text for the
discussion of the language-independent analysis of the problems from his book
that we discuss.
We have also made these programs self-contained, with a {\tt main()}  that
does something useful.  In this, we tend to go a little beyond what is 
in Quinn's examples, mostly in order to show how the serial side of \Xten{} might be
used to organize the melange of trivia that seems necessary to implement any real
application.

We highly recommend having Quinn's book close at hand when working through these
advanced examples.  Quinn's treatment of the MPI and OpenMP
approaches to these problems will be very helpful, particularly if you have not
had a lot of experience with this sort of programming.  
\Xten, with the advantage of many years of hindsight into the good and bad
of parallel language design, sometimes has more
graceful ways of saying what needs to be said, but a lot of insight went into the
design of MPI and OpenMP.  Becoming fluent in them is not only its own reward, but
it will help in understanding where \Xten{} is coming from.
}

\section*{Trademarks}
The following is a list of the trademarked names that appear in the text:
\begin{description}
\item[Mac OS] registered trademark of Apple Computer Inc.
\item[Linux]  registered trademark of Linus Torvalds.
\item[Unix] registered trademark licensed through X/Open Company, Ltd
\item[Windows] a shorthand for one of a number of operating systems,
Windows XP, Windows Vista, and Windows 7, that are registered trademarks of
Microsoft Corporation.
\end{description}

\section*{Getting X10 Onto Your Machine}

We've put together some installation guides for various platforms that show in
detail how to set up your workstation to develop \Xten{} programs. The
platforms supported are
\begin{description}
\item[Mac:]\href{http://dist.codehaus.org/x10/documentation/install/mac-cl-install.html} 
{http://dist.codehaus.org/x10/documentation/install/mac-cl-install.html}
\item[Linux:]\href{http://dist.codehaus.org/x10/documentation/install/linux-cl-install.html} 
{http://dist.codehaus.org/x10/documentation/install/linux-cl-install.html}
\item[Windows:]\href{http://dist.codehaus.org/x10/documentation/install/windows-cl-install.html}
{http://dist.codehaus.org/x10/documentation/install/windows-cl-install.html}
\end{description}

Once you have the environment set up, you will be able to run the programs in
this guide.

For Windows users, we have also developed a batch installation of the Cygwin
environment that \Xten requires.  This will make it easier to get started than
getting what you need piecemeal off the Web. The installer can be downloaded
from
\begin{quote}
\href{http://dist.codehaus.org/x10/documentation/install/cygwin-install.zip}
{http://dist.codehaus.org/x10/documentation/install/cygwin-install.zip}
\end{quote}
You can refer to the Windows installation guide above for more details about its
use.

If you are an Eclipse user, the \Xten{} Development Toolkit (X10DT) is a
plugin that can be installed
into Eclipse 3.5.x (Galileo) or Eclipse 3.6.x (Helios).  If you are unfamiliar
with Eclipse, it is a free Open Source IDE, and we recommend you take a look at
its introductory Web page, \href{http://www.eclipse.org/home/newcomers.php}
{http://www.eclipse.org/home/newcomers.php}.  
If you have Eclipse installed, you can go to the 
{\tt Help} menu and select ``{\tt Install New Software}''.  You will be asked
to add an update site.  The url to use for X10DT is:
\begin{quote}
\href{http://dist.codehaus.org/x10/x10dt/2.1/updateSite/}
{http://dist.codehaus.org/x10/x10dt/2.1/updateSite/}.
\end{quote}
If you want to install both Eclipse and X10DT together from scratch, there is
an all-in-one installation that contains everything you need to start
building X10 programs.  It can be downloaded from the \Xten{} web site:
\begin{quote}
\href{http://x10.codehaus.org/X10DT+2.1+Installation}
{http://x10.codehaus.org/X10DT+2.1+Installation}.
\end{quote}

The \Xten{} web site, \href{http://x10.codehaus.org/}
{http://x10.codehaus.org/}, is a source for the latest news about
\Xten{}. 
The \Xten{} mailing lists are all hosted at \href{http://x10.codehaus.org/For+Users}
{http://x10.codehaus.org/For+Users}. 
You can subscribe to any of them, but the ones most likely 
to be of interest are
\begin{itemize}
\item x10-announce: for (infrequent) announcements of \Xten{} releases, and
\item x10-users: for a general \Xten{} user list.
\end{itemize}
\section*{Acknowledgements}
Particular thanks go to David Grove and Igor Peshansky, who have gone out of their
way to answer our questions and to lay out the rationale behind \Xten's design.  Rachel
Bellamy, Vijay Saraswat, Janice Shepherd, Mandana Vaziri, and Yoav Zibin were
kind enough to read a early versions carefully, and the guide is much the
better for their comments.
